\documentclass[a4paper,11pt]{article}
\usepackage[T1]{fontenc}
\usepackage[utf8]{inputenc}
\usepackage{lmodern}
\usepackage{hyperref}
\usepackage{graphicx}
\usepackage{rotating}
\usepackage{listings}
\usepackage{color}

% Swedish
\usepackage[swedish]{babel}

% Table of contents depth 3 levels: A.B.C
\setcounter{tocdepth}{3}

\begin{document}

\title{Sommarstugekoll \\
	Digital Konstruktion EDA234 \\ Grupp 2}
\author{Fredrik Brosser, Karl Buchka, Andreas Henriksson, Johan Wolgers \\
   Chalmers Tekniska Högskola \\
   \texttt{email addresses}}

\maketitle

\pagebreak

\tableofcontents

\pagebreak

\begin{abstract}

	Systemet som beskrivs i denna rapport är ett automatiserat temperaturkontrollsystem som är åtkomstbart och styrbart via
	telefon (DTMF). Systemets huvudfunktionalitet är att informera användaren om de aktuella temperaturerna som läses från
	två anslutna temperatursensorer, och att ge användaren möjlighet att på avstånd styra (av/på) ett antal externa funktioner,
	såsom exempelvis värmeelement. Information och kontrolldata utbytes genom en vanlig analog telefonlinje, med hjälp av DTMF:
	Dual Tone Multiple Frequency. 

	Ett särskilt exempel på användningsområde är kontroll av uppvärmningen i en sommarstuga. Med hjälp av systemet kan man
	undvika att temperaturen sjunker under en viss gräns, vilket skulle kunna få konsekvenser såsom att rör fryser. Genom att
	hålla temperaturen på en lämplig nivå kan man undvika oönskade konsekvenser samtidigt som man minimerar uppvärmningskostnader.
	'Sommarstugekoll' löser detta problem genom att erbjuda ett enkelt sätt att hålla koll på och kontrollera inom- och utomhustemperaturer.
	Om ägaren till en sommarstuga vill göra ett besök ringer han/hon helt enkelt upp sommarstugan och slår på lämpliga element, så att
	sommarstugan hinner värmas till en behaglig temperatur medans ägaren kör dit.

	Systemet självt kräver endast en telefonlinje och en vanlig spänningskälla på 5V. Eftersom det är effektsnålt kan systemet stå
	på under långa perioder, vilket gör det en lätt och trevlig lösning på alla dina sommarstugeuppvärmningsproblem.

\end{abstract}

\begin{abstract}

	The system described in this report is an automated domestic temperature control system, accessible and controllable via
	a standard telephone connection (DTMF). The main functionality of the system is informing the user of the current temperatures
	at the points of measurement, and also to give the user the ability to remotely control (simple on/off) a number of external
	functions. These functions could be, for example, heating systems, radiators or air conditioning. Information- and control data
	is exchanged via a standard, analog telephone connection, using DTMF: Dual Tone, Multiple Frequency.

	A specific, practical usage example is the concerned holiday home owner wanting to keep control of the heating in his summer house.
	Using Sommarstugekoll, one can keep the temperature at a reasonable level, avoiding for example pipes freezing, but at the same time
	keeping the total heating costs to a minimum. Sommarstugekoll solves this by offering users a neat and simple way to keep track
	of in- and outdoor temperatures - when the previously mentioned summer house owner wishes to make a visit, he simply dials the
	summer house telephone number giving it instructions to bring the temperature up to a comfortable level, all while he makes
	the drive up. Simple - smart - elegant - Sommarstugekoll is the domestic temperature automation assistant of the future!

\end{abstract}


\pagebreak

\section{Introduktion}

	Konstruktionsprojektet har utförts inom ramarna för kursen "Digital Konstruktion EDA234" vid Chalmers Tekniska Högskola.
	Uppgiften var att inom gruppen (4 personer) relativt självständigt (med stöd från handledare) konstruera och dokumentera
	ett digital system utifrån en vag specifikation, där fokus låg på huvudfunktionaliteten.
	Utvecklingen har skett med hjälp av ett färdigt baslaborationskort som sedan byggts på med externa kretsar för den funktionalitet
	som specifierats. Logikkretsen som har använts är en Xilinx XC9572XL CPLD, och utveckligen har i huvudsak skett i VHDL och Xilinx ISE-miljön,
	samt i ModelSim för simuleringar. Denna rapport är uppdelad i flera olika abstraktionslager för att passa läsaren och dennes specifika intresseområden. 

	En ren användarmanual finns bifogad i Appendix.

\section{Systemspecifikation}

	\begin{tabular}{ l r}
	   Matningsspänning & +5V\\
	   Strömförbrukning & ~80mA\\
	   Någonting & Någonting\\
	\end{tabular}

\section{Systembeskrivning}

	\subsection{Uppdelning}

	Systemet består av ett antal delblock (moduler), och är skrivet för att vara så modulärt som möjligt.
	Nedan följer en sammanfattande tabell över de olika modulerna och deras funktioner. Varje modul är mer
	utförligt beskriven i sektionen {\it Block, Funktionalitet}. För en komplett överblick, se Blockschema.

	\begin{tabular}{ l r}
		{\bf Modul} & {\bf Funktion}\\
	   	Styrenhet & Samordnar systemfunktioner\\
	  	Temperaturmodul & Initierar, läser och presenterar temperatur\\
	   	DTMF-Modul & Tar emot DTMF-signaler från användaren\\
		Ljud-Modul & Spelar upp ljud lagrade på extern minneskrets\\
		Knappsats-Modul & Hanterar knapptryckningar från användaren\\
		Funktions-Modul & Hanterar funktionerna och deras status\\
	\end{tabular}

	Systemet är uppdelat enligt dataväg-styrenhet-modellen, där designprincipen går ut på att skilja styrsignaler
	och dataflödeskontroll (Styrenheten) från själva datan som forslas genom systemet (Datavägen). Den data som
	skickas inom systemet är i första hand temperaturdata från temperaturmodulen till ljudmodulen, och ljud som
	lagras på den externa ljudlagringskretsen och hanteras av ljudmodulen. Detta flöde kontrolleras från styrenheten via
	enkla styrsignaler. Viss data, i form av indata från användargränssnitt och utdata till funktionsstatus, passerar
	dock genom styrenheten.

	\subsection{Översikt}

	Systemet bygger på DTMF-kommunikation (Dual Tone Multiple Frequency, Tonval). Telefonen skickar ut DTMF-signaler, som kan
	läsas och avkodas av DTMF-avkodarkretsen, MT8880. DTMF-signaler består av två toner som tillsammans unikt ger vilken
	siffra på telefonens knappsats som användaren tryckt på. 
 

\section{Blockschema}

	\begin{figure}[h!]
	  \centering
	      \includegraphics[scale=0.48, angle=90]{BlockDiagramCPLD1.png}
	  	\caption{Blockschema (CPLD1)}
	\end{figure}

	\begin{figure}[h!]
	  \centering
	      \includegraphics[scale=0.48, angle=90]{BlockDiagramCPLD2.png}
	  	\caption{Blockschema (CPLD2)}
	\end{figure}

\section{Block, Funktionalitet}

	\subsection{Dataväg}

		

	\subsection{Styrenhet}

	\subsection{Funktionsmoduler}

		\subsubsection{DTMF-Modul och MT8880}
	
		\subsubsection{Ljudmodul och ISD2560P}

		\subsubsection{Temperaturmodul och DS18S20}

{\bf Syfte}

Temperaturmodulen i CPLD'n är ansvarig för att hantera seriekommunikationen med 
DS18S20-temperatursensorerna, via entrådsbussarna, samt att ge ut de lästa temperaturerna
i tecken-belopp-format till ljudmodulen, i syfte att låta den i sin tur spela upp de avlästa
temperaturen för systemanvändaren.\\

{\noindent \bf 1-Trådsbuss}

Entrådsbussarna är anslutna till matningsspänning (+5V) genom ett pull-up-motstånd på 4.7kO, och
ändarna av bussen är anslutna till mastern (CPLD) och sensorn (DS18S20), respektive. Då bussen
befinner sig i viloläge dras den hög ("svag drivning") av pull-up-motståndet. När information
skickas över bussen drar den kommunicerande (sändande) enheten bussen låg genom att driva den
med en stark logisk nolla.\\

	\begin{figure}[h!tb]
	  \centering
	      \includegraphics[scale=1, angle=0]{TempBus.png}
	  	\caption{Uppkoppling, Entrådsbuss}
	\end{figure}

{\noindent \bf DS18S20}

Den temperatursensor som används är Maxim DS18S20, som ger temperaturmodulen möjlighet att läsa av temperaturen
med nio (9) bitars upplösning. Sensorerna som användas i detta system drivs av en extern spänningskälla på +5V,
och kommunicerar seriellt över en entrådsbuss. Seriekommunikationen baserar på att bus-mastern initierar skriv-
och läs-luckor. Varje sådan lucka är mellan 60 och 120 us lång. Temperatursensorn initieras genom att mastern
driver bussen låg i åtminstone 480 us, vilket följs av att temperatursensorkretsen själv drar bussen låg i 60-240 us,
efter en återhämtningslucka på minst 1 us. Efter initieringen väntar sensorkretsen på ett ROM-kommando från mastern,
följt av ett Funktions-kommando. Varje sådant kommando är en byte lång, och skickas som LSB-först. Mastern skickar
en logisk nolla genom att driva bussen låg under hela skrivluckan. En etta skickas genom att mastern driver bussen låg
under en kort period, 1-15us, och sedan släpper bussen under resten av skrivluckans längd. Mellan varje skriv- eller läslucka
måste det finnas en återhämtningsperiod på minst 1us.

Då mastern är klar med att skicka över ROM- och Funktions-kommando, kan (beroende på vilka kommandon som sändes) DS18S20-kretsen
svara med aktuell data. På samma sätt som ovan måste mastern här initiera en läs-lucka genom att driva bussen låg i 1-8us, och
sedan släppa den (högimpediv). DS18S20-kretsen svarar på den allokerade läs-luckan genom att antingen hålla bussen låg för att
överföra en nolla, eller genom att låta bussen dras hög av pull-up-motståndet för en etta. Under den här tiden (upp till 15 us
efter att ha släppt bussen) kan mastern sampla bussen för att läsa av vad DS18S20-kretsen skickat. All data som skickas från
temperatursensorn skickas som LSB-först, och i 2-komplementsform.\\

{\noindent \bf Läscykel, Sammanfattning}

En läscykel består av fyra steg: Initialisering, Kommandon, Läsning och Viloläge.

Initialiseringen består i av att mastern driver bussen låg i 512 us, sedan släpper den.
Temperatursensorn svarar med en närvaro-puls genom att driva bussen låg i 106 us, och därgenom bekräftar den sin närvaro
på bussen och sin operationella status.

Då mastern detekterat närvaropulsen börjar den överföra ett ROM-kommando (Skip ROM, 0xCC), följt av en kort återhämtningslucka
och sedan ett Funktions-kommando (Convert Temperature, 0x44), enligt läs-/skriv-luckemetoden som beskrivits ovan.
Skip-ROM-kommandot används i det här systemet eftersom endast en temperatursensor används per entrådsbuss.
Därmed finns inget behov av att kunna addressera specifika sensorer på bussen.
Convert Temperature-kommandot säger till DS18S20-kretsen att börja konvertera temperaturen, och sedan spara den 
lästa temperaturen i sitt interna minne för att sedan läsas av mastern.
Under konverteringsperioden (upp till 750 ms) kan mastern polla temperatursensorns status genom att kontinuerligt
sända förfrågningar genom att dra bussen låg i en kort period (4 us). Temperatursensorn svarar med en nolla
sålänge den är upptagen med att konvertera temperatur, och sedan en etta såfort den är klar.
Då mastern ser att temperatursensorn är klar, initiseras temperatursensorn om och ytterligare ett ROM-Funktions-kommandopar
överförs. Dessa är 0xCC (Skip ROM) följt av 0xBE (Read Scratchpad), respektive.
Idealt ska temperatursensorkretsen nu vara redo att överföra temperaturdata till mastern från sitt interna 9-bytesminne.
Då den ges Read Scratchpad-kommandot börjar DS18S20 sända över innehållet i sitt minne över bussen.
Mastern går då in i läsningsläget och börjar sampla datan på bussen genom att driva bussen låg, släppa den och sampla efter 4 us,
allt enligt den metod som beskrivs ovan. Mastern samplar de första åtta bitarna data som sänds av DS18S20, sedan en ytterligare,
nionde bit. Den sista biten utgör en tecken-bit, 0 för positiv temperatur och 1 för negativ. Då den nionde biten data lästs
ger mastern på nytt en initieringspuls, som säger åt temperatursensorn att sluta sända data.
Den nyligen lästa temperaturen placeras av temperaturmodulen i CPLD'n (på CPLD2) på den interna temperaturbussen, och
TAv sätts till 1 för att indikera att det finns korrekt data på bussen, enligt vad som efterfrågades av styrenheten.\\

	\begin{figure}[h!tb]
	  \centering
	      \includegraphics[scale=0.5, angle=0]{ReadCycleFlowChart.png}
	  	\caption{Högnivå-flödesdiagram för Temperaturläscykel}
	\end{figure}

{\noindent \bf Signaler}

	\begin{tabular}{l r}
		\\{\bf In} &  \\
		DQ0 & Kommunikationstråd till temperatursensor 0\\
		DQ1 & Kommunikationstråd till temperatursensor 1\\
		TRd & Styrsignal från styrenheten: Signalerar att läscykel ska inledas\\
		TSel & Styrsignal från styrenheten som väljer temperatursensor (0/1)\\\\
		{\bf Ut} &  \\
		TAv & Signal till styrenheten för att signalera att valid data finns på bussen\\
		Temp & Den interna temperaturdatabussen till ljudmodulen\\\\
	\end{tabular}

{\noindent \bf Uppbyggnad}

Temperaturmodulen är baserad runt en tillståndsmaskin, understödd av en intern Buffer/MUX-modul samt 
ett antal interna räknare för att generera de nödvändiga tidsfördröjningspulserna.\\

{\noindent \bf Buffer/MUX}

Buffer/MUX-modulen är direkt ansluten till de två DS18S20-Temperatursensorerna som använder entrådsbusskommunikationen.
Multiplexern (MUX) används för att välja mellan vilken av de två sensorerna (Sesnor 0 eller Sensor 1) som styrenheten
vill kommunicera med, och använder signalen TSel.
Buffern är en tri-state-buffer med en Enable-signal, E. När E är satt till 1 så kan mastern använda den av TSel valda
entrådsbussen som en utgång för att överföra kommandon. När E är satt till 0 kan mastern läsa data från bussen.\\

{\noindent \bf Räknare}

Internt använder temperaturmodulen ett antal olika räknare: \\
	\begin{tabular}{l c r}
		\\{\bf Namn} & {\bf Bitar} & {\bf Syfte}\\
		cntInt & 9 & Genererar timing-pulser\\
		ZC & 4 & Hanterar timing för skriv-luckor\\
		Progress & 2 & Anger aktuellt kommando (0-3)\\
		bitCnt & 8 & Anger aktuell bit för överföring/läsning\\\\
	\end{tabular}

{\noindent \bf Tillståndsmaskin}

För en uttömmande grafisk beskrivning av den interna tillståndsmaskinen som används av temperaturmodulen,
se sektionen om tillståndsmaskiner.\\

		\subsubsection{Kontrollfunktioner}

		\subsubsection{Gränssnitt och Knappar}

\section{Tillståndsmaskiner}
		\subsubsection{Styrenhet}
		\subsubsection{Temperaturmodul}
			\\(Refererar till tillståndsdiagramet för temperaturmodulen)\\
			{\bf Reset:} Alla signaler och räknare återställs till sina grundvärden.\\
			{\bf Grundvärden:} Utgångspunkten är att alla signaler behåller sina gamla värden om inget annat anges.\\
			\begin{tabular}{l}
				\\{\bf Viloläge (Tillstånd 0):}\\
				0: Viloläge och återställningspunkt. Inväntar TRd = 1 från Styrenheten\\
				{\bf Initialisering (Tillstånd 1-3):}\\
				1: Fördröjningstillstånd, väntar på puls på DelayLong (512 us)\\
				2: Mastern driver bussen låg i 512 us och sätter datavärdet till 0xCC (Skip ROM)\\
				3: Mastern släpper bussen, DS18S20 skickar närvaropuls\\
				{\bf Kommandoöverföring (Tillstånd 4-8):}\\
				4: Förberedelsetillstånd före sändning\\
				5: Huvudsändningstillståndet, mastern sänder bit enligt aktuellt räknarvärde\\
				6: Mellanbittillstånd, återhämtningstillständ. Fortsätter om fler bitar ska skickas\\
				7: Avgör om DS18S20 är upptagen med att konvertera temperaturen. Om så, vänta tills klar\\
				8: Vägskälstillstånd. Om vi har fler kommandon att skicka, gå tillbaka, annars börja läsa\\
				{\bf Läsning (Tillstånd 9-15):}\\
				9:  Förberedelsetillstånd före läsning\\
				10: Mastern initierar en läslucka genom att dra bussen låg i 4 us\\
				11: Mastern väntar ytterligare 4 us för att pull-up-motståndet ska få verka\\
				12: Mastern samplar bussen. Om DS18S20 skickar en nolla hålls bussen låg, annars inte\\
				13: Återhämtningstillstånd mellan samplingar\\
				14: Vägskälstillstånd. Om vi har fler bitar att läsa, gå tillbaka, annars gå vidare\\
				15: Läsning klar, lägg ut temperatur på buss och signallera valid data. Gå tillbaka till 0\\
			\end{tabular}

	\begin{figure}[h!tb]
	  \centering
	      \includegraphics[scale=0.5, angle=0]{StateMachineExplained.png}
	  	\caption{Förklaring till State Machine-diagram}
	\end{figure}

	\begin{figure}[h!tb]
	  \centering
	      \includegraphics[scale=0.4, angle=0]{TempStateMachineDiagram.png}
	  	\caption{Detaljerat State Machine-diagram för Temperaturläscykel}
	\end{figure}

		\subsubsection{DTMF-Modul}
		\subsubsection{Ljudmodul}

\section{Tidsdiagram}

	\begin{figure}[h!tb]
	  \centering
	      \includegraphics[scale=0.5, angle=0]{TempTiming.png}
	  	\caption{Timingdiagram för Temperaturmodulen}
	\end{figure}

\section{Felanalys}

\section{Appendix}

	\subsection{Komponentlista}
	\begin{tabular}{l r}
		\\{\bf Namn} & {\bf Beskrivning}\\
		XC9572XL (x2) & CPLD\\
		DS18S20 (x2) & Temperatursensor\\
		MT8880C & DTMF Transceiver\\
		ISD2560P & Ljudlagringskrets\\\\
	\end{tabular}

	\subsection{Kretsschema}

	\subsection{Pin-Layout}

		\begin{figure}[h!]
		  \centering
		      \includegraphics[scale=0.48, angle=0]{PinDiagram.png}
		  	\caption{Pin-Konfiguration (CPLD1)}
		\end{figure}

		\begin{figure}[h!]
		  \centering
		      \includegraphics[scale=0.48, angle=0]{PinDiagram.png}
		  	\caption{Pin-Konfiguration (CPLD2)}
		\end{figure}

	\subsection{Signallista}
	\begin{tabular}{l c1 c2 r}
		\\{\bf Signal} & {\bf Typ} & {\bf Från} & {\bf Till} & {\bf Beskrivning}\\ \\
		clk & Global & Extern & Allt & Global Klocksignal\\
		rst & Global & Extern & Allt & Global Resetsignal\\
		Buttons[3..0] & Databuss & Knappsats & Knappsatsmodul & Indatavektor\\
		KBDav & Available & Knappsats & Knappsatsmodul & Availablesignal\\
		FuncData[3..0] & Databuss & Funktionsmodul & FunktionsLEDs & Funktionsstyrningssignaler\\
		DTMFData[3..0] & Databuss & DTMF-Modul & MT8880 & Databuss till MT8880\\
		Phi2 & Klocksignal & DTMF-Modul & MT8880 & Klocksignal till MT8880\\
		R/W & Styrsignal & DTMF-Modul & MT8880 & Read/Write till MT8880\\
		RS0 & Styrsignal & DTMF-Modul & MT8880 & Intieringssignal till MT8880\\
		IRQ & Interrupt & MT8880 & DTMF-Modul & Interruptsignal från MT8880\\

		KData[3..0] & Databuss & Knappsatsmodul & Styrenhet & Data från knappsatsmodul\\
		KAv & Styrsignal & Knappsatsmodul & Styrenhet & Availablesignal från knappsatsmodul\\
		KAck & Acknowledgement & Styrenhet & Knappsatsmodul & Acknowledgementsignal till knappsatsmodul\\

		TSel & Databuss & Styrenhet & Temperaturmodul & Selectsignal för temperatursensorer (0/1)\\
		TRd & Styrsignal & Styrenhet & Temperaturmodul & Read-signal (starta läsning)\\
		TAv & Available & Temperaturmodul & Styrenhet & Availablesignal, valid data på buss\\

		SData[3..0] & Adressbuss & Styrenhet & Ljudmodul & Ljudadressbuss\\
		SSel & Styrsignal & Styrenhet & Ljudmodul & Selectsignal för temperatur/ljud\\
		SPlay & Styrsignal & Styrenhet & Ljudmodul & Play-signal (spela ljud)\\
		SDone & Styrsignal & Ljudmodul & Styrenhet & Signallerar uppspelning färdig\\

		DData[3..0] & Databuss (bidir.) & DTMF-Modul & Styrenhet & DTMF-Databuss\\
		DAv & Acknowledgement & DTMF-Modul & Styrenhet & Availablesignal från DTMF-Modul\\
		DAck & Styrsignal & Styrenhet & DTMF-Modul & Acknowledgement från Styrenhet\\

		FData[3..0] & Databuss & Styrenhet & Funktionsmodul & Funktionsstatusbuss\\

		SP+ & Analog & ISD250P & MT8880 & Analog ljudsignal (+)\\
		SP- & Analog & ISD250P & MT8880 & Analog ljudsignal (-)\\

		DQ0 & Seriellbuss (bidir.) & Temperaturmodul & DS18S20 & 1-trådsbuss\\
		DQ1 & Seriellbuss (bidir.) & Temperaturmodul & DS18S20 & 1-trådsbuss\\

		Temp[7..0] & Databuss & Temperaturmodul & Ljudmodul & Temperaturdatabuss\\

		Addr[4..0] & Adressbuss & Ljudmodul & ISD2560P & Adressbuss till ISD2560P\\
		CE & Styrsignal & Ljudmodul & ISD2560P & Chip-Enable från ljudmodul\\
		EOM & Interrupt & IDS2560P & Ljudmodul & End-Of-Message från ISD2560P\\\\
	\end{tabular}

	\subsection{Programlistningar}
	
	\subsection{Arbetsfördelning}

	\begin{tabular}{l c r}
		\\{\bf Namn} & {\bf Moduler / Ansvar} & {\bf Övrigt}\\
		Fredrik Brosser 	& ?? 	& ??\\
		Karl Buchka 		& ?? 	& ??\\
		Andreas Henriksson 	& ?? 	& ??\\
		Johan Wolgers 		& ??	& ??\\\\
	\end{tabular}

\end{document}
